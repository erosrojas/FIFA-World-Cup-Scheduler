%% Standard start of a latex document
\documentclass[letterpaper,12pt]{article}
%% Always use 12pt - it is much easier to read
%% Things written after '%' sign, are ignored by the latex editor - they are how to introduce comments into your .tex source
%% Anything mathematics related should be put in between '$' signs.

%% Set some names and numbers here so we can use them below
\newcommand{\myname}{Group 1} %%%%%%%%%%%%%%% ---------> Change this to your name
% \newcommand{\mynumber}{12345678} %%%%%%%%%%%%%%% ---------> Change this to your student number
% \newcommand{\hw}{1} %%%%%%%%%%%%%%% --------->  set this to the homework number

%%%%%%
%% There is a bit of stuff below which you should not have to change
%%%%%%

%% AMS mathematics packages - they contain many useful fonts and symbols.
\usepackage{amsmath, amsfonts, amssymb}

%% The geometry package changes the margins to use more of the page, I suggest
%% using it because standard latex margins are chosen for articles and letters,
%% not homework.
\usepackage[paper=letterpaper,left=25mm,right=25mm,top=3cm,bottom=25mm]{geometry}
%% For details of how this package work, google the ``latex geometry documentation''.

%%
%% Fancy headers and footers - make the document look nice
\usepackage{fancyhdr} %% for details on how this work, search-engine ``fancyhdr documentation''
\pagestyle{fancy}
%%
%% The header
\lhead{MATH 441} % course name as top-left
\chead{Group Project: Proposal} % homework number in top-centre
\rhead{ \myname }
%% This is a little more complicated because we have used `` \\ '' to force a line-break between the name and number.
%%
%% The footer
\lfoot{\myname} % name on bottom-left
\cfoot{Page \thepage} % page in middle
%\rfoot{\mynumber} % student number on bottom-right
%%
%% These put horizontal lines between the main text and header and footer.
\renewcommand{\headrulewidth}{0.4pt}
\renewcommand{\footrulewidth}{0.4pt}
%%%


% Some useful macros

\usepackage{amsmath,amssymb,amsthm}
\usepackage{enumerate}

\newcommand{\ZZ}{\mathbb{Z}}
\newcommand{\FF}{\mathbb{F}}
\newcommand{\RR}{\mathbb{R}}
\newcommand{\QQ}{\mathbb{Q}}
\newcommand{\CC}{\mathbb{C}}
\newcommand{\NN}{\mathbb{N}}
\renewcommand\vec{\mathbf}

\newcommand{\st}{\text{ such that } }
\newcommand{\dee}[1]{\mathrm{d}#1}
\newcommand{\diff}[2]{ \frac{\dee{#1}}{\dee{#2}} }
\newcommand{\lt}{<}
\newcommand{\gt}{>}
\newcommand{\set}[1]{\left\{#1 \right\}}


\begin{document}
% Put your answsers as items in this enumerate environment and they will be automatically numbered
\subsection*{Group Project: Proposal}
\medskip

\subsubsection*{Problem Statement: } Optimize scheduling for the group stages of the Fifa 2026 world cup.
\begin{itemize}
\item Schedule games around regions and cities, where each team gets an optimal amount of rest and minimal travel time, while adhering to the given constraints.

\item Utilize combinatorial optimization or constraint satisfaction programming to figure out scheduling, matching teams to cities, minimizing travel time for each team and the various other tournament constraints.

\item We will initially start off by defining a structure to represent a possible match-up, and represent all match-ups on a tree, where a node at depth ‘i’ would be a match on day ‘i’. Then we could use constraint satisfaction to backtrack and search for a path where all nodes satisfy the given constraints.
\end{itemize}

\subsubsection*{Data + Computations: } 
Data: 
\begin{itemize}
\itemsep0em 
\item Previous world cup (Fifa 2022) data.
\item Use current ranking of top 48 teams to determine which teams are potentially within this years world cup
\item Use pre-existing FIFA world cup draws to design our random draw of groups
\end{itemize}
Constraints:
\begin{itemize}
\itemsep0em 
\item 48 Teams total
\item 12 Groups (4 teams each group)
\item 3 Regions (16 teams per region, teams only stay within their own region)
\item 3 Games per team (play each other team within the group once)
\item 3 Days of rest between every game for each team (affects travel time and scheduling)
\item Ensure that most popular teams play within the bigger stadiums (define using previous world cup data + existing stadium capacity data + viewership/pay-per-view data)
\end{itemize}
\subsubsection*{References + Examples: } 
\begin{itemize}
\itemsep0em 
\item Travelling salesman problem (for scheduling)
\item Linear programming (for discrete optimization within constraints)
\item World tour scheduling problem (for inspiration)
\item Previous world cup data + viewership data + stadium capacity data
\end{itemize}

\end{document}
